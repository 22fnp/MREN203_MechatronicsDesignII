% QU-HANDOUT template for LaTeX
% qu-handout-example.tex
% Originally created by Joshua A. Marshall on 11 July 2013

% Set document parameters (font, paper, eqn position, page style)
\documentclass[letterpaper,fleqn,oneside]{template}

% ---------------------------------------------------------------

% Some handy commands ... add your own!
\newcommand{\norm}[1]{\left\Vert#1\right\Vert}
\newcommand{\abs}[1]{\left\vert#1\right\vert}
\newcommand{\set}[1]{\left\{#1\right\}}
\newcommand{\Real}{\mathbb R}
\newcommand{\Complex}{\mathbb C}
\newcommand{\eps}{\varepsilon}
\newcommand{\To}{\longrightarrow}
\newcommand{\Ker}{\textup{Ker}}
\newcommand{\Img}{\textup{Img}}
\newcommand{\diag}{\textup{diag}}
\newcommand{\circulant}{\textup{circ}}
\newcommand{\mbf}{\mathbf}
\newcommand*\dif{\mathop{}\!\mathrm{d}}

\pretolerance=1000  % Avoid breaking words too easily
\tolerance=2000     % Allow some flexibility before hyphenating
\emergencystretch=1em  % Extra slack to adjust spacing without hyphenating

% ---------------------------------------------------------------

% ---------------------------------------------------------------

\begin{document}

% Set the page style for the document
\pagestyle{plain}

% ---------------------------------------------------------------

% Title Page
\begin{titlepage}
    \begin{center}
        \vspace*{1.2cm}  
        \Huge
        \textbf{Letter of Intent}\\
        \vspace{0.5cm}
        \large
        \vspace{2cm}
        Prepared by Group 10\\
        \vspace{0.8cm}
        \textbf{Sebastien Sauter, Marlow Gaddes, and Jamie Strain}\\ 
        % Insert a figure
\begin{figure}
  \begin{center}
    \includegraphics[width=\textwidth]{LaTex Template/qu-logo-horizontal-colour.pdf}
    %\includegraphics[width=4in]{figs/qu-logo-vertical-colour.pdf}
    %\includegraphics[height=4in]{figs/qu-logo-vertical-colour.pdf}
    %\includegraphics[width=0.5\textwidth]{figs/qu-logo-vertical-colour.pdf}
    \caption{This is an example figure.}
    \label{fig:example}
  \end{center}
\end{figure}
        \vfill
        \small
        Non-Binding Initial Proposal for Here and Now Delivery\\
        \vspace{0.8cm}
        MREN 203 - Mechatronics and Robotics Design II\\
        \vspace{0.8cm}
        \textbf{January 25, 2025} 
    \end{center}
\end{titlepage}

% ---------------------------------------------------------------

% Main body is below
\section{About SSG Robotics}



\section{Problem Definition}


\section{Conceptual Overview}

In response to HND, SSG Robotics proposes a semi-autonomous mobile robot equipped to handle deliveries of goods in an airport environment upon user request.\\

A simple overview of the sequence of events is as follows: The user (passenger or staff) requests and pays for a delivery service using a web application. The robot processes the request in a priority queue and executes the next task. Depending on the order, the system alerts a representative from the retailer to either begin making the order or to set aside the required goods. The robot proceeds to determine the optimal route to the pickup point via its path finding algorithm and navigates itself to that point with the use of its Sharp IR and LiDAR sensors. The retail representative deposits the goods into a latched compartment on the robot that is then locked. The robot once again calculates its optimal route and travels to users’ delivery point, making use of its horn or lights if necessary. The user unlocks the compartment with a custom unlocking technique and takes the goods they ordered. The user receives an estimated time of arrival calculated from the position of the robot throughout the duration of the order. For feasibility this process must occur more efficiently or faster than what the user could have done on their own.

\subsection{Solution Requirements}

To execute the proposed solution SSG Robotics has outlined five key requirements that must be met during the design and prototyping phase. 

\begin{enumerate}
    \item \textbf{Time Constraint:} Must make deliveries before the user has to board their plane or in a similar amount of time that the user would do the task themselves.
    \item \textbf{Navigation:} Must determine the global optimal route and then navigate a dynamically changing local environment.
    \item \textbf{User Interface:} An application or website must allow the user to make the order, pay for their order and view or track the status of their order.
    \item \textbf{Safety:} The contents of the order cannot be tampered, stolen or pose any security threat to the airport.
    \item \textbf{Error Rate:} Little to no error is permitted.
\end{enumerate}

\subsection{Solution Assumptions}

A list of reasonable assumptions has been prepared for HND. Assume:
\begin{itemize}
    \item The entire terminal of the airport is wheelchair accessible.
    \item Retail representatives are trained to interact with the robot and can retrieve or prepare orders in a timeline manner. 
    \item Retailers should be able to process online orders or be notified ahead of time. 
    \item The user does not leave the terminal of the airport during the order. 
    \item The user takes all the contents they ordered from the robot. 
    \item  The robot is not physically disabled by other people in the airport. 
\end{itemize}

The applications of the robot are not limited to small deliveries to flight passengers. Future considerations for this technology can include transporting high value items from gate to gate for airlines, transportation of small carry-on luggage from gate to gate in time sensitive situations and potentially assisting visitors with directions to their gate or to baggage claim.






\section{Strategic Plan}

\textbf{Table \ref{tbl:deliverables}}.

\begin{table}[htbp]
  \caption{Project Deliverables}
  \begin{center}
    \begin{tabular}{lrc}
      \toprule
      \bf Deliverable & \bf Submission \\ \midrule
      Letter of Intent & Week 2 \\
      Technical Update 1 & Week 4 \\
      Conceptual Design Report & Week 7 \\
      Technical Update 2 & Week 10 \\
      Final Design Exhibit & Week 12 \\
      \bottomrule
    \end{tabular}
  \end{center}
  \label{tbl:deliverables}
\end{table}%




The above block might be better as a figure.  There are also lots of ``algorithm'' packages for LaTeX, which you might use for writing pseudo-code.  Notice in the previous sentence how we write quotes in \LaTeX.

% ---------------------------------------------------------------

\end{document}
