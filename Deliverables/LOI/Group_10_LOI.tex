% QU-HANDOUT template for LaTeX
% qu-handout-example.tex
% Originally created by Joshua A. Marshall on 11 July 2013

% Set document parameters (font, paper, eqn position, page style)
\documentclass[letterpaper,fleqn,oneside]{template}

% ---------------------------------------------------------------

% Some handy commands ... add your own!
\newcommand{\norm}[1]{\left\Vert#1\right\Vert}
\newcommand{\abs}[1]{\left\vert#1\right\vert}
\newcommand{\set}[1]{\left\{#1\right\}}
\newcommand{\Real}{\mathbb R}
\newcommand{\Complex}{\mathbb C}
\newcommand{\eps}{\varepsilon}
\newcommand{\To}{\longrightarrow}
\newcommand{\Ker}{\textup{Ker}}
\newcommand{\Img}{\textup{Img}}
\newcommand{\diag}{\textup{diag}}
\newcommand{\circulant}{\textup{circ}}
\newcommand{\mbf}{\mathbf}
\newcommand*\dif{\mathop{}\!\mathrm{d}}

\pretolerance=1000  % Avoid breaking words too easily
\tolerance=2000     % Allow some flexibility before hyphenating
\emergencystretch=1em  % Extra slack to adjust spacing without hyphenating



% ---------------------------------------------------------------

% ---------------------------------------------------------------

\begin{document}
\title{Letter of Intent - SSG Robotics}
% Set the page style for the document
\pagestyle{plain}

% ---------------------------------------------------------------

% Title Page
\begin{titlepage}
    \begin{center}
        \vspace*{1.2cm}  
        \Huge
        \textbf{Letter of Intent}\\
        
        \vspace{0.5cm}
        \large
        \vspace{2cm}
        Prepared by Group 10\\
        
        \vspace{0.8cm}
        \textbf{Sebastien Sauter, Marlow Gaddes, and Jamie Strain}\\ 
        
        \vspace{1.5cm} % Add vertical space for logo placement
        \includegraphics[width=0.4\textwidth]{Deliverables/LOI/SSG_Robotics_Logo.png} % Adjust width as needed
        
        \vfill
        \small
        Non-Binding Initial Proposal for Here and Now Delivery\\
        
        \vspace{0.8cm}
        MREN 203 - Mechatronics and Robotics Design II\\
        
        \vspace{0.8cm}
        \textbf{January 25, 2025} 
    \end{center}
\end{titlepage}

% ---------------------------------------------------------------

% Main body is below
\section{About SSG Robotics}



\section{Problem Definition}
SSG Robotics objective is to develop a mobile robot that will navigate and deliver goods within airports to both travelers and employees. This environment was chosen as a good fit for this problem since airports are often crowded and cover a large area. This bot will allow for time-sensitive deliveries and goods to move through the airport more efficiently and without human personnel being used, reducing costs for the airport. For travelers, this will also relieve the stress of navigating the airport to get things they may need before their flights. The bot will not require any human intervention once sent on delivery and will navigate using both a sensor and a built-in navigation system to find the optimal route. 

\section{Conceptual Overview}

In response to HND, SSG Robotics proposes a semi-autonomous mobile robot equipped to handle deliveries of goods in an airport environment upon user request.\\

A simple overview of the sequence of events is as follows: The user (passenger or staff) requests and pays for a delivery service using a web application. The robot processes the request in a priority queue and executes the next task. Depending on the order, the system alerts a representative from the retailer to either begin making the order or to set aside the required goods. The robot proceeds to determine the optimal route to the pickup point via its path finding algorithm and navigates itself to that point with the use of its Sharp IR and LiDAR sensors. The retail representative deposits the goods into a latched compartment on the robot that is then locked. The robot once again calculates its optimal route and travels to users’ delivery point, making use of its horn or lights if necessary. The user unlocks the compartment with a custom unlocking technique and takes the goods they ordered. The user receives an estimated time of arrival calculated from the position of the robot throughout the duration of the order. For feasibility this process must occur more efficiently or faster than what the user could have done on their own.

\subsection{Solution Requirements}

To execute the proposed solution SSG Robotics has outlined five key requirements that must be met during the design and prototyping phase. 

\begin{enumerate}
    \item \textbf{Time Constraint:} Must make deliveries before the user has to board their plane or in a similar amount of time that the user would do the task themselves.
    \item \textbf{Navigation:} Must determine the global optimal route and then navigate a dynamically changing local environment.
    \item \textbf{User Interface:} An application or website must allow the user to make the order, pay for their order and view or track the status of their order.
    \item \textbf{Safety:} The contents of the order cannot be tampered, stolen or pose any security threat to the airport.
    \item \textbf{Reliability:} Little to no error is permitted.
\end{enumerate}

\subsection{Solution Assumptions}

A list of reasonable assumptions has been prepared for HND. Assume:
\begin{itemize}
    \item The entire terminal of the airport is wheelchair accessible.
    \item Retail representatives are trained to interact with the robot and can retrieve or prepare orders in a timeline manner. 
    \item Retailers should be able to process online orders or be notified ahead of time. 
    \item The user does not leave the terminal of the airport during the order. 
    \item The user takes all the contents they ordered from the robot. 
    \item  The robot is not physically disabled by other people in the airport. 
\end{itemize}

The applications of the robot are not limited to small deliveries to flight passengers. Future considerations for this technology can include transporting high value items from gate to gate for airlines, transportation of small carry-on luggage from gate to gate in time sensitive situations and potentially assisting visitors with directions to their gate or to baggage claim.

\section{Strategic Plan}


\subsection{Project Leads}
\begin{itemize}
    \item Jamie Strain - Software Lead
    \item Sebastien Sauter - Electrical and Project Management Lead
    \item Marlow Gaddes - Mechanical and Documentation Lead
\end{itemize}
\subsection{Team Standards and Expectations}
\begin{itemize}
    \item Communication: iMessages and Email
    \item File Storage: GitLab and OneDrive
    \item Deliverables: LaTeX or MS Word
    \item Weekly Workshops: Mondays, 8:30-10:30 am and Tuesdays, 9:30-11:30 am.
    \item Weekly Meetings: Thursdays, 2:30-3:30 pm. 
\end{itemize}



\subsection{Project Timeline}

Starting from the release of the RFP, the project has been given a 12-week timeline. SSG Robotics is expected to submit 5 deliverables summarized by \textbf{Table \ref{tbl:deliverables}}.



\begin{table}[htbp]
  \caption{Project Deliverables}
  \begin{center}
    \begin{tabular}{llc}
      \toprule
      \bf Deliverable & \bf Description & \bf Submission Date \\ \midrule
      Letter of Intent & Non-Binding Initial Proposal & Week 2 \\
      Technical Update 1 & Prototype Development Summary & Week 4 \\
      Conceptual Design Report & Final Design Proposal & Week 7 \\
      Technical Update 2 & Prototype Development Summary & Week 10 \\
      Final Design Exhibit & Public Prototype Showcase & Week 12 \\
      \bottomrule
    \end{tabular}
  \end{center}
  \label{tbl:deliverables}
\end{table}%

Technical challenges and delays are expected while developing the prototype. Therefore, prototype development is to be front-loaded during the workshops to identify design challenges early. Deliverables are to be discussed during the weekly meetings and completed independently. 


% ---------------------------------------------------------------

\end{document}
